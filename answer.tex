\documentclass{article} %article 文档
% \usepackage{ctex}  %使用宏包(为了能够显示汉字)
\title{COMP4670 Assignment 1 Theory Answers}  %文章标题
\author{Jieli Zheng}   %作者的名称
\date{u6579712}
% 设置页面的环境,a4纸张大小,左右上下边距信息
\usepackage[a4paper,left=10mm,right=10mm,top=15mm,bottom=15mm]{geometry}
\usepackage{array}  
\usepackage{multirow}
\usepackage{booktabs}
\usepackage{graphicx}
\usepackage{float} 
\usepackage{subfigure}
\usepackage{amsmath}
\usepackage{color}
\usepackage{ulem}
\usepackage{subfig}

%行间距
\usepackage{setspace}
\renewcommand{\baselinestretch}{1.2}

\begin{document}
\maketitle

\section*{Exercise 1  Properties of Independent Variables}

\subsection*{1}
By definition of expectation of a continuous random variable:\\
\begin{align*}
    \mathbf{E}_{X} &= \int_{\Omega} X(\omega)dP(\omega), where\, x,y,x+y\in\Omega \\
\end{align*}
Therefore, \\
\begin{align*}
    \mathbf{E}_{X,Y}[x+y] &= \iint (x+y)f(x,y) dxdy \text{, f is the probability density function of x,y}\\
    &= \iint xf(x,y) dxdy +\iint yf(x,y) dxdy \\
    &= \mathbf{E}_{X}[x]+\mathbf{E}_{Y}[y] \quad \text{Q.E.D.}\\
\end{align*}


\subsection*{2}
By definition of expectation of a continuous random variable:\\
\begin{align*}
    \mathbf{E}_{X} &= \int_{\Omega} X(\omega)dP(\omega), where\, x,y,xy\in\Omega \\
\end{align*}
Therefore, \\
\begin{align*}
    \mathbf{E}_{X,Y}[xy] = \iint xy f(x,y) dxdy \text{, f is the probability density function of x,y}\\
    \text{Because X and Y are independent from each other, we have}\quad
    f(x,y)  &= f(x)f(y)\\
\end{align*}
Then we have \\
\begin{align*}
 \mathbf{E}_{X,Y}[xy] &= \iint xyf(x)f(y)dxdy\\
 &= \int xf(x)dx\int yf(y)dy\\
 &= \mathbf{E}_{X}\mathbf{E}_{Y} \quad \text Q.E.D.\\
\end{align*}

\subsection*{3}
By definition of convariance:\\
\begin{align*}
    cov[x,y] &= \mathbf{E}_{X,Y}[x-\mathbf{E}_{X}[x])(y-\mathbf{E}_{Y}[y])]\\
\end{align*}
Therefore, \\
\begin{align*}
     cov[x,y] &= \mathbf{E}[xy-y\mathbf{E}_{X}[x]-x\mathbf{E}_{Y}[y]+\mathbf{E}_{X}\mathbf{E}_{Y}]\\ 
\end{align*}
From the proof mentioned in 1.1,1.2:\\
\begin{align*}
 \mathbf{E}_{X,Y}[xy]&=\mathbf{E}_{X}\mathbf{E}_{Y} \quad \text (1)\\
\mathbf{E}_{X,Y}[x+y] &= \mathbf{E}_{X}[x]+\mathbf{E}_{Y}[y] \quad \text (2)\\
\end{align*}
we have 
\begin{align*}
 cov[x,y] &= \mathbf{E}[xy]-\mathbf{E}[\mathbf{E}_{X}[x]y]-\mathbf{E}[\mathbf{E}_{Y}[y]x]+\mathbf{E}[\mathbf{E}_{X}\mathbf{E}_{Y}]\\
 &= \mathbf{E}_{X,Y} - \mathbf{E}_{X}\mathbf{E}_{Y} -\mathbf{E}_{Y}\mathbf{E}_{X} + \mathbf{E}_{X,Y} \\
 &= \mathbf{E}_{X,Y}-  \mathbf{E}_{X}\mathbf{E}_{Y}\\
 &=0  \quad \text Q.E.D.\\ 
\end{align*}


\section*{Exercise 2  Beta Priors}
\subsection*{1}
\begin{align*}
    B(a,b) &= \int_{0}^{1} x^{a-1} (1-x)^{b-1} dx \text{, where } a,b \geq 1\\
    \text{Use Integration by substitution,}\\
    &=\dfrac{1}{a}\int_{0}^{1}(1-x)^{b-1} d(x^a)\\
    \text{Use Integration by parts,}\\
    &=(x^{a-1}(1-x)^{b-1})\Big|_{0}^{1}-\dfrac{1}{b}\int_{0}^{1} x^{a-1}d(1-x)^b\\
    &=0 + \dfrac{1}{b}\int_{0}^{1} x^{a-1} dx^b                                \\
    &=\int_{0}^{1} x^{a-1}x^{b-1}dx\\
    &=\int_{0}^{1} x^{a+b-2}dx\\
    &=\dfrac{1}{a+b-1} x^{a+b-1}\Big|_{0}^{1}\\
    &=\dfrac{1}{a+b-1}(1-0)\\
    &a+b-1 \geq 1\text{, given } a,b\geq 1\\
    &\text{Therefore, } \dfrac{1}{a+b-1}\in (0,1] \quad \text{Q.E.D.}\\
\end{align*}

\subsection*{2}
    First of all ,to prove $Beta(x|a,b)$ is a valid probability distribution, we 
    need to prove:\\
    \begin{align*}
        &(1)\quad \forall x\in[0,1], Beta(x|a,b) > 0\\
        &(2)\quad \int_{0}^{1}Beta(x|a,b) = 1\\
    \end{align*}
    Proof of (1):\\
    \begin{align*}
        &Beta(x|a,b) =\dfrac{1}{B(a,b)}x^{a-1}(1-x)^{b-1}\\
        &\text{Also from 2.1, } B(a,b) = \dfrac{1}{a+b-1} \\ 
        &\text{Therefore, } Beta(x|a,b) =(a+b-1)x^{a-1}(1-x)^{b-1}\\
        &\text{Since } a,b \geq 1\\
        &\text{Then }a+b-1 \geq 1 \geq 0\\
        &\text{Because } x\in[0,1]\\
        &\forall m \in \mathcal{R} \quad x^{m}>0   \\
        &\text{(except for 0, where }0^0 \text{ can be defined as 1 or remain undefined)}\\
        &\text{Hence } x^{a-1}>0,\,(1-x)^{b-1} >0\\
        &\text{Therefore, }Beta(x|a,b) = (a+b-1)x^{a-1}(1-x)^{b-1} \geq 0 
        \text{ because 3 multipliers are all non-negative. }\\
    \end{align*}
    Proof of (2):\\
    According to Exercise 2.1,
    \begin{align*} 
    \int_{0}^{1}Beta(x|a,b) &= \dfrac{1}{B(a,b)}\int_{0}^{1}x^{a-1}(1-x)^{b-1}dx\\
    &\text{ because }\dfrac{1}{B(a,b)}\text{ is a constant according to 2.1}\\
    &= \dfrac{1}{a+b-1}* (a+b-1)\\
    &=1\\
    \end{align*}
    Therefore, the integral of PDF $Beta(x|a,b)$ is equal to 1.\\
    Hence this PDF is a valid probability distribution.\\\\

    Secondly, we need to prove $Beta(x|a,b)$ is well defined.\\
    Because $B(a,b)$ (which is the denominator) is larger than 1, the only potential singular point for $Beta(x|a,b)$ is when a = 1, x = 0 or b = 1, x = 1 which makes a $0^0$.\\
    However, usually $0^0$ can be defined as 1.\\
    Hence this PDF is a well defined probability distribution.\\

    From the proofs above, we can say that $Beta(x|a,b)$ is a well defined and valid probability distribution.\\


\subsection*{3}
    By definition of Gamma function and Beta function by Gamma function,
\begin{align*}
    \text{LEFT} &= \dfrac{\Gamma(a+1)\Gamma(b)}{\Gamma(a+b+1)}\\
    &= \dfrac{a\Gamma(a)\Gamma(b)}{(a+b)\Gamma(a+b)}\\
    &= \dfrac{a}{a+b}\dfrac{\Gamma(a)\Gamma(b)}{\Gamma(a+b)}\\
    &=\text{RIGHT} \quad \text{Q.E.D.}\\
\end{align*}

\section*{Exercise 3  Coin Flips in Generality}
\subsection*{1}


\section*{Exercise 4  Noisy Coin Flips}



\end{document}